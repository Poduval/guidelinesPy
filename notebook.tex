
% Default to the notebook output style

    


% Inherit from the specified cell style.




    
\documentclass[11pt]{article}

    
    
    \usepackage[T1]{fontenc}
    % Nicer default font (+ math font) than Computer Modern for most use cases
    \usepackage{mathpazo}

    % Basic figure setup, for now with no caption control since it's done
    % automatically by Pandoc (which extracts ![](path) syntax from Markdown).
    \usepackage{graphicx}
    % We will generate all images so they have a width \maxwidth. This means
    % that they will get their normal width if they fit onto the page, but
    % are scaled down if they would overflow the margins.
    \makeatletter
    \def\maxwidth{\ifdim\Gin@nat@width>\linewidth\linewidth
    \else\Gin@nat@width\fi}
    \makeatother
    \let\Oldincludegraphics\includegraphics
    % Set max figure width to be 80% of text width, for now hardcoded.
    \renewcommand{\includegraphics}[1]{\Oldincludegraphics[width=.8\maxwidth]{#1}}
    % Ensure that by default, figures have no caption (until we provide a
    % proper Figure object with a Caption API and a way to capture that
    % in the conversion process - todo).
    \usepackage{caption}
    \DeclareCaptionLabelFormat{nolabel}{}
    \captionsetup{labelformat=nolabel}

    \usepackage{adjustbox} % Used to constrain images to a maximum size 
    \usepackage{xcolor} % Allow colors to be defined
    \usepackage{enumerate} % Needed for markdown enumerations to work
    \usepackage{geometry} % Used to adjust the document margins
    \usepackage{amsmath} % Equations
    \usepackage{amssymb} % Equations
    \usepackage{textcomp} % defines textquotesingle
    % Hack from http://tex.stackexchange.com/a/47451/13684:
    \AtBeginDocument{%
        \def\PYZsq{\textquotesingle}% Upright quotes in Pygmentized code
    }
    \usepackage{upquote} % Upright quotes for verbatim code
    \usepackage{eurosym} % defines \euro
    \usepackage[mathletters]{ucs} % Extended unicode (utf-8) support
    \usepackage[utf8x]{inputenc} % Allow utf-8 characters in the tex document
    \usepackage{fancyvrb} % verbatim replacement that allows latex
    \usepackage{grffile} % extends the file name processing of package graphics 
                         % to support a larger range 
    % The hyperref package gives us a pdf with properly built
    % internal navigation ('pdf bookmarks' for the table of contents,
    % internal cross-reference links, web links for URLs, etc.)
    \usepackage{hyperref}
    \usepackage{longtable} % longtable support required by pandoc >1.10
    \usepackage{booktabs}  % table support for pandoc > 1.12.2
    \usepackage[inline]{enumitem} % IRkernel/repr support (it uses the enumerate* environment)
    \usepackage[normalem]{ulem} % ulem is needed to support strikethroughs (\sout)
                                % normalem makes italics be italics, not underlines
    

    
    
    % Colors for the hyperref package
    \definecolor{urlcolor}{rgb}{0,.145,.698}
    \definecolor{linkcolor}{rgb}{.71,0.21,0.01}
    \definecolor{citecolor}{rgb}{.12,.54,.11}

    % ANSI colors
    \definecolor{ansi-black}{HTML}{3E424D}
    \definecolor{ansi-black-intense}{HTML}{282C36}
    \definecolor{ansi-red}{HTML}{E75C58}
    \definecolor{ansi-red-intense}{HTML}{B22B31}
    \definecolor{ansi-green}{HTML}{00A250}
    \definecolor{ansi-green-intense}{HTML}{007427}
    \definecolor{ansi-yellow}{HTML}{DDB62B}
    \definecolor{ansi-yellow-intense}{HTML}{B27D12}
    \definecolor{ansi-blue}{HTML}{208FFB}
    \definecolor{ansi-blue-intense}{HTML}{0065CA}
    \definecolor{ansi-magenta}{HTML}{D160C4}
    \definecolor{ansi-magenta-intense}{HTML}{A03196}
    \definecolor{ansi-cyan}{HTML}{60C6C8}
    \definecolor{ansi-cyan-intense}{HTML}{258F8F}
    \definecolor{ansi-white}{HTML}{C5C1B4}
    \definecolor{ansi-white-intense}{HTML}{A1A6B2}

    % commands and environments needed by pandoc snippets
    % extracted from the output of `pandoc -s`
    \providecommand{\tightlist}{%
      \setlength{\itemsep}{0pt}\setlength{\parskip}{0pt}}
    \DefineVerbatimEnvironment{Highlighting}{Verbatim}{commandchars=\\\{\}}
    % Add ',fontsize=\small' for more characters per line
    \newenvironment{Shaded}{}{}
    \newcommand{\KeywordTok}[1]{\textcolor[rgb]{0.00,0.44,0.13}{\textbf{{#1}}}}
    \newcommand{\DataTypeTok}[1]{\textcolor[rgb]{0.56,0.13,0.00}{{#1}}}
    \newcommand{\DecValTok}[1]{\textcolor[rgb]{0.25,0.63,0.44}{{#1}}}
    \newcommand{\BaseNTok}[1]{\textcolor[rgb]{0.25,0.63,0.44}{{#1}}}
    \newcommand{\FloatTok}[1]{\textcolor[rgb]{0.25,0.63,0.44}{{#1}}}
    \newcommand{\CharTok}[1]{\textcolor[rgb]{0.25,0.44,0.63}{{#1}}}
    \newcommand{\StringTok}[1]{\textcolor[rgb]{0.25,0.44,0.63}{{#1}}}
    \newcommand{\CommentTok}[1]{\textcolor[rgb]{0.38,0.63,0.69}{\textit{{#1}}}}
    \newcommand{\OtherTok}[1]{\textcolor[rgb]{0.00,0.44,0.13}{{#1}}}
    \newcommand{\AlertTok}[1]{\textcolor[rgb]{1.00,0.00,0.00}{\textbf{{#1}}}}
    \newcommand{\FunctionTok}[1]{\textcolor[rgb]{0.02,0.16,0.49}{{#1}}}
    \newcommand{\RegionMarkerTok}[1]{{#1}}
    \newcommand{\ErrorTok}[1]{\textcolor[rgb]{1.00,0.00,0.00}{\textbf{{#1}}}}
    \newcommand{\NormalTok}[1]{{#1}}
    
    % Additional commands for more recent versions of Pandoc
    \newcommand{\ConstantTok}[1]{\textcolor[rgb]{0.53,0.00,0.00}{{#1}}}
    \newcommand{\SpecialCharTok}[1]{\textcolor[rgb]{0.25,0.44,0.63}{{#1}}}
    \newcommand{\VerbatimStringTok}[1]{\textcolor[rgb]{0.25,0.44,0.63}{{#1}}}
    \newcommand{\SpecialStringTok}[1]{\textcolor[rgb]{0.73,0.40,0.53}{{#1}}}
    \newcommand{\ImportTok}[1]{{#1}}
    \newcommand{\DocumentationTok}[1]{\textcolor[rgb]{0.73,0.13,0.13}{\textit{{#1}}}}
    \newcommand{\AnnotationTok}[1]{\textcolor[rgb]{0.38,0.63,0.69}{\textbf{\textit{{#1}}}}}
    \newcommand{\CommentVarTok}[1]{\textcolor[rgb]{0.38,0.63,0.69}{\textbf{\textit{{#1}}}}}
    \newcommand{\VariableTok}[1]{\textcolor[rgb]{0.10,0.09,0.49}{{#1}}}
    \newcommand{\ControlFlowTok}[1]{\textcolor[rgb]{0.00,0.44,0.13}{\textbf{{#1}}}}
    \newcommand{\OperatorTok}[1]{\textcolor[rgb]{0.40,0.40,0.40}{{#1}}}
    \newcommand{\BuiltInTok}[1]{{#1}}
    \newcommand{\ExtensionTok}[1]{{#1}}
    \newcommand{\PreprocessorTok}[1]{\textcolor[rgb]{0.74,0.48,0.00}{{#1}}}
    \newcommand{\AttributeTok}[1]{\textcolor[rgb]{0.49,0.56,0.16}{{#1}}}
    \newcommand{\InformationTok}[1]{\textcolor[rgb]{0.38,0.63,0.69}{\textbf{\textit{{#1}}}}}
    \newcommand{\WarningTok}[1]{\textcolor[rgb]{0.38,0.63,0.69}{\textbf{\textit{{#1}}}}}
    
    
    % Define a nice break command that doesn't care if a line doesn't already
    % exist.
    \def\br{\hspace*{\fill} \\* }
    % Math Jax compatability definitions
    \def\gt{>}
    \def\lt{<}
    % Document parameters
    \title{Python for statistics}
    
    
    

    % Pygments definitions
    
\makeatletter
\def\PY@reset{\let\PY@it=\relax \let\PY@bf=\relax%
    \let\PY@ul=\relax \let\PY@tc=\relax%
    \let\PY@bc=\relax \let\PY@ff=\relax}
\def\PY@tok#1{\csname PY@tok@#1\endcsname}
\def\PY@toks#1+{\ifx\relax#1\empty\else%
    \PY@tok{#1}\expandafter\PY@toks\fi}
\def\PY@do#1{\PY@bc{\PY@tc{\PY@ul{%
    \PY@it{\PY@bf{\PY@ff{#1}}}}}}}
\def\PY#1#2{\PY@reset\PY@toks#1+\relax+\PY@do{#2}}

\expandafter\def\csname PY@tok@w\endcsname{\def\PY@tc##1{\textcolor[rgb]{0.73,0.73,0.73}{##1}}}
\expandafter\def\csname PY@tok@c\endcsname{\let\PY@it=\textit\def\PY@tc##1{\textcolor[rgb]{0.25,0.50,0.50}{##1}}}
\expandafter\def\csname PY@tok@cp\endcsname{\def\PY@tc##1{\textcolor[rgb]{0.74,0.48,0.00}{##1}}}
\expandafter\def\csname PY@tok@k\endcsname{\let\PY@bf=\textbf\def\PY@tc##1{\textcolor[rgb]{0.00,0.50,0.00}{##1}}}
\expandafter\def\csname PY@tok@kp\endcsname{\def\PY@tc##1{\textcolor[rgb]{0.00,0.50,0.00}{##1}}}
\expandafter\def\csname PY@tok@kt\endcsname{\def\PY@tc##1{\textcolor[rgb]{0.69,0.00,0.25}{##1}}}
\expandafter\def\csname PY@tok@o\endcsname{\def\PY@tc##1{\textcolor[rgb]{0.40,0.40,0.40}{##1}}}
\expandafter\def\csname PY@tok@ow\endcsname{\let\PY@bf=\textbf\def\PY@tc##1{\textcolor[rgb]{0.67,0.13,1.00}{##1}}}
\expandafter\def\csname PY@tok@nb\endcsname{\def\PY@tc##1{\textcolor[rgb]{0.00,0.50,0.00}{##1}}}
\expandafter\def\csname PY@tok@nf\endcsname{\def\PY@tc##1{\textcolor[rgb]{0.00,0.00,1.00}{##1}}}
\expandafter\def\csname PY@tok@nc\endcsname{\let\PY@bf=\textbf\def\PY@tc##1{\textcolor[rgb]{0.00,0.00,1.00}{##1}}}
\expandafter\def\csname PY@tok@nn\endcsname{\let\PY@bf=\textbf\def\PY@tc##1{\textcolor[rgb]{0.00,0.00,1.00}{##1}}}
\expandafter\def\csname PY@tok@ne\endcsname{\let\PY@bf=\textbf\def\PY@tc##1{\textcolor[rgb]{0.82,0.25,0.23}{##1}}}
\expandafter\def\csname PY@tok@nv\endcsname{\def\PY@tc##1{\textcolor[rgb]{0.10,0.09,0.49}{##1}}}
\expandafter\def\csname PY@tok@no\endcsname{\def\PY@tc##1{\textcolor[rgb]{0.53,0.00,0.00}{##1}}}
\expandafter\def\csname PY@tok@nl\endcsname{\def\PY@tc##1{\textcolor[rgb]{0.63,0.63,0.00}{##1}}}
\expandafter\def\csname PY@tok@ni\endcsname{\let\PY@bf=\textbf\def\PY@tc##1{\textcolor[rgb]{0.60,0.60,0.60}{##1}}}
\expandafter\def\csname PY@tok@na\endcsname{\def\PY@tc##1{\textcolor[rgb]{0.49,0.56,0.16}{##1}}}
\expandafter\def\csname PY@tok@nt\endcsname{\let\PY@bf=\textbf\def\PY@tc##1{\textcolor[rgb]{0.00,0.50,0.00}{##1}}}
\expandafter\def\csname PY@tok@nd\endcsname{\def\PY@tc##1{\textcolor[rgb]{0.67,0.13,1.00}{##1}}}
\expandafter\def\csname PY@tok@s\endcsname{\def\PY@tc##1{\textcolor[rgb]{0.73,0.13,0.13}{##1}}}
\expandafter\def\csname PY@tok@sd\endcsname{\let\PY@it=\textit\def\PY@tc##1{\textcolor[rgb]{0.73,0.13,0.13}{##1}}}
\expandafter\def\csname PY@tok@si\endcsname{\let\PY@bf=\textbf\def\PY@tc##1{\textcolor[rgb]{0.73,0.40,0.53}{##1}}}
\expandafter\def\csname PY@tok@se\endcsname{\let\PY@bf=\textbf\def\PY@tc##1{\textcolor[rgb]{0.73,0.40,0.13}{##1}}}
\expandafter\def\csname PY@tok@sr\endcsname{\def\PY@tc##1{\textcolor[rgb]{0.73,0.40,0.53}{##1}}}
\expandafter\def\csname PY@tok@ss\endcsname{\def\PY@tc##1{\textcolor[rgb]{0.10,0.09,0.49}{##1}}}
\expandafter\def\csname PY@tok@sx\endcsname{\def\PY@tc##1{\textcolor[rgb]{0.00,0.50,0.00}{##1}}}
\expandafter\def\csname PY@tok@m\endcsname{\def\PY@tc##1{\textcolor[rgb]{0.40,0.40,0.40}{##1}}}
\expandafter\def\csname PY@tok@gh\endcsname{\let\PY@bf=\textbf\def\PY@tc##1{\textcolor[rgb]{0.00,0.00,0.50}{##1}}}
\expandafter\def\csname PY@tok@gu\endcsname{\let\PY@bf=\textbf\def\PY@tc##1{\textcolor[rgb]{0.50,0.00,0.50}{##1}}}
\expandafter\def\csname PY@tok@gd\endcsname{\def\PY@tc##1{\textcolor[rgb]{0.63,0.00,0.00}{##1}}}
\expandafter\def\csname PY@tok@gi\endcsname{\def\PY@tc##1{\textcolor[rgb]{0.00,0.63,0.00}{##1}}}
\expandafter\def\csname PY@tok@gr\endcsname{\def\PY@tc##1{\textcolor[rgb]{1.00,0.00,0.00}{##1}}}
\expandafter\def\csname PY@tok@ge\endcsname{\let\PY@it=\textit}
\expandafter\def\csname PY@tok@gs\endcsname{\let\PY@bf=\textbf}
\expandafter\def\csname PY@tok@gp\endcsname{\let\PY@bf=\textbf\def\PY@tc##1{\textcolor[rgb]{0.00,0.00,0.50}{##1}}}
\expandafter\def\csname PY@tok@go\endcsname{\def\PY@tc##1{\textcolor[rgb]{0.53,0.53,0.53}{##1}}}
\expandafter\def\csname PY@tok@gt\endcsname{\def\PY@tc##1{\textcolor[rgb]{0.00,0.27,0.87}{##1}}}
\expandafter\def\csname PY@tok@err\endcsname{\def\PY@bc##1{\setlength{\fboxsep}{0pt}\fcolorbox[rgb]{1.00,0.00,0.00}{1,1,1}{\strut ##1}}}
\expandafter\def\csname PY@tok@kc\endcsname{\let\PY@bf=\textbf\def\PY@tc##1{\textcolor[rgb]{0.00,0.50,0.00}{##1}}}
\expandafter\def\csname PY@tok@kd\endcsname{\let\PY@bf=\textbf\def\PY@tc##1{\textcolor[rgb]{0.00,0.50,0.00}{##1}}}
\expandafter\def\csname PY@tok@kn\endcsname{\let\PY@bf=\textbf\def\PY@tc##1{\textcolor[rgb]{0.00,0.50,0.00}{##1}}}
\expandafter\def\csname PY@tok@kr\endcsname{\let\PY@bf=\textbf\def\PY@tc##1{\textcolor[rgb]{0.00,0.50,0.00}{##1}}}
\expandafter\def\csname PY@tok@bp\endcsname{\def\PY@tc##1{\textcolor[rgb]{0.00,0.50,0.00}{##1}}}
\expandafter\def\csname PY@tok@fm\endcsname{\def\PY@tc##1{\textcolor[rgb]{0.00,0.00,1.00}{##1}}}
\expandafter\def\csname PY@tok@vc\endcsname{\def\PY@tc##1{\textcolor[rgb]{0.10,0.09,0.49}{##1}}}
\expandafter\def\csname PY@tok@vg\endcsname{\def\PY@tc##1{\textcolor[rgb]{0.10,0.09,0.49}{##1}}}
\expandafter\def\csname PY@tok@vi\endcsname{\def\PY@tc##1{\textcolor[rgb]{0.10,0.09,0.49}{##1}}}
\expandafter\def\csname PY@tok@vm\endcsname{\def\PY@tc##1{\textcolor[rgb]{0.10,0.09,0.49}{##1}}}
\expandafter\def\csname PY@tok@sa\endcsname{\def\PY@tc##1{\textcolor[rgb]{0.73,0.13,0.13}{##1}}}
\expandafter\def\csname PY@tok@sb\endcsname{\def\PY@tc##1{\textcolor[rgb]{0.73,0.13,0.13}{##1}}}
\expandafter\def\csname PY@tok@sc\endcsname{\def\PY@tc##1{\textcolor[rgb]{0.73,0.13,0.13}{##1}}}
\expandafter\def\csname PY@tok@dl\endcsname{\def\PY@tc##1{\textcolor[rgb]{0.73,0.13,0.13}{##1}}}
\expandafter\def\csname PY@tok@s2\endcsname{\def\PY@tc##1{\textcolor[rgb]{0.73,0.13,0.13}{##1}}}
\expandafter\def\csname PY@tok@sh\endcsname{\def\PY@tc##1{\textcolor[rgb]{0.73,0.13,0.13}{##1}}}
\expandafter\def\csname PY@tok@s1\endcsname{\def\PY@tc##1{\textcolor[rgb]{0.73,0.13,0.13}{##1}}}
\expandafter\def\csname PY@tok@mb\endcsname{\def\PY@tc##1{\textcolor[rgb]{0.40,0.40,0.40}{##1}}}
\expandafter\def\csname PY@tok@mf\endcsname{\def\PY@tc##1{\textcolor[rgb]{0.40,0.40,0.40}{##1}}}
\expandafter\def\csname PY@tok@mh\endcsname{\def\PY@tc##1{\textcolor[rgb]{0.40,0.40,0.40}{##1}}}
\expandafter\def\csname PY@tok@mi\endcsname{\def\PY@tc##1{\textcolor[rgb]{0.40,0.40,0.40}{##1}}}
\expandafter\def\csname PY@tok@il\endcsname{\def\PY@tc##1{\textcolor[rgb]{0.40,0.40,0.40}{##1}}}
\expandafter\def\csname PY@tok@mo\endcsname{\def\PY@tc##1{\textcolor[rgb]{0.40,0.40,0.40}{##1}}}
\expandafter\def\csname PY@tok@ch\endcsname{\let\PY@it=\textit\def\PY@tc##1{\textcolor[rgb]{0.25,0.50,0.50}{##1}}}
\expandafter\def\csname PY@tok@cm\endcsname{\let\PY@it=\textit\def\PY@tc##1{\textcolor[rgb]{0.25,0.50,0.50}{##1}}}
\expandafter\def\csname PY@tok@cpf\endcsname{\let\PY@it=\textit\def\PY@tc##1{\textcolor[rgb]{0.25,0.50,0.50}{##1}}}
\expandafter\def\csname PY@tok@c1\endcsname{\let\PY@it=\textit\def\PY@tc##1{\textcolor[rgb]{0.25,0.50,0.50}{##1}}}
\expandafter\def\csname PY@tok@cs\endcsname{\let\PY@it=\textit\def\PY@tc##1{\textcolor[rgb]{0.25,0.50,0.50}{##1}}}

\def\PYZbs{\char`\\}
\def\PYZus{\char`\_}
\def\PYZob{\char`\{}
\def\PYZcb{\char`\}}
\def\PYZca{\char`\^}
\def\PYZam{\char`\&}
\def\PYZlt{\char`\<}
\def\PYZgt{\char`\>}
\def\PYZsh{\char`\#}
\def\PYZpc{\char`\%}
\def\PYZdl{\char`\$}
\def\PYZhy{\char`\-}
\def\PYZsq{\char`\'}
\def\PYZdq{\char`\"}
\def\PYZti{\char`\~}
% for compatibility with earlier versions
\def\PYZat{@}
\def\PYZlb{[}
\def\PYZrb{]}
\makeatother


    % Exact colors from NB
    \definecolor{incolor}{rgb}{0.0, 0.0, 0.5}
    \definecolor{outcolor}{rgb}{0.545, 0.0, 0.0}



    
    % Prevent overflowing lines due to hard-to-break entities
    \sloppy 
    % Setup hyperref package
    \hypersetup{
      breaklinks=true,  % so long urls are correctly broken across lines
      colorlinks=true,
      urlcolor=urlcolor,
      linkcolor=linkcolor,
      citecolor=citecolor,
      }
    % Slightly bigger margins than the latex defaults
    
    \geometry{verbose,tmargin=1in,bmargin=1in,lmargin=1in,rmargin=1in}
    
    

    \begin{document}
    
    
    \maketitle
    
    

    
    \section{Python basics}\label{python-basics}

Python is an interpreted, high-level, \textbf{general-purpose}
programming language. It can be easy to pick up whether you're a first
time programmer or you're experienced with other languages. Please find
below some useful links.

\url{https://www.python.org}

\url{https://docs.python.org/3.6/index.html}

\url{https://jupyter.org/}

    \subsection{Types of objects}\label{types-of-objects}

\begin{itemize}
\tightlist
\item
  \textbf{Integers (int):} whole numbers, e.g. \texttt{2\ 200\ 20000}
  etc.
\item
  \textbf{Floating point (float):} numbers with a decimal point, e.g.
  \texttt{2.3\ 200.50\ 2000.56} etc.
\item
  \textbf{Strings (str):} ordered sequence of characters, e.g.
  \texttt{"hello","Rakesh","2000"} etc.
\item
  \textbf{Lists (list):} ordered sequence of objects, e.g.
  \texttt{{[}2,"hello",2.2{]}}
\item
  \textbf{Dictionaries (dict):} unordered key-value pairs, e.g.
  \texttt{\{"my\_key":"key\_value",\ "first\_name":"Rakesh"\}}
\item
  \textbf{Tuples (tup):} ordered immutable sequence of objects, e.g.
  \texttt{(2,"hello",2.2)}
\item
  \textbf{Sets (set):} unordered collection of unique objects, e.g.
  \texttt{("a","b")}\\
\item
  \textbf{Booleans (bool):} logical value indicating \texttt{True}, or
  \texttt{False}
\end{itemize}

    \subsection{Variable assignments}\label{variable-assignments}

We use a single \texttt{=} sign to assign values/labels to variables
(\texttt{variable\_name\ =\ object}). The names you use when creating
these labels need to follow a few rules:

\begin{itemize}
\tightlist
\item
  Names can't start with a number.
\item
  Spaces are not allowed in the name, use \texttt{\_} instead.
\item
  Can't use any of the following symbols -
  \texttt{\textquotesingle{}",\textless{}\textgreater{}/?\textbar{}\textbackslash{}()!@\#\$\%\^{}\&*\textasciitilde{}-+}.
\item
  The best practice to name them is in \texttt{lowercase}.
\item
  Avoid using the characters \texttt{l,\ O\ or\ I} as single character
  variable names.
\item
  Avoid using words that have special meaning in Python like
  \texttt{list} and \texttt{str}.
\end{itemize}

Python uses \emph{dynamic typing}, meaning you can reassign variables to
different data types. This makes Python very flexible in assigning data
types and speed up developement time. Note that this may result in
unexpected bugs, you need to be aware of. You can check what type of
object is assigned to a variable using Python's built-in \texttt{type()}
function. Common data types include

    \subsection{Math operations - (int \&
float)}\label{math-operations---int-float}

Basically these are performed on integers and floating points.

    \begin{Verbatim}[commandchars=\\\{\}]
{\color{incolor}In [{\color{incolor} }]:} \PY{n+nb}{print}\PY{p}{(}\PY{l+m+mi}{2}\PY{o}{+}\PY{l+m+mi}{1}\PY{p}{)} \PY{c+c1}{\PYZsh{} Addition}
        \PY{n+nb}{print}\PY{p}{(}\PY{l+m+mi}{2}\PY{o}{\PYZhy{}}\PY{l+m+mi}{1}\PY{p}{)} \PY{c+c1}{\PYZsh{} Subtraction}
        \PY{n+nb}{print}\PY{p}{(}\PY{l+m+mi}{2}\PY{o}{*}\PY{l+m+mi}{2}\PY{p}{)} \PY{c+c1}{\PYZsh{} Multiplication}
        \PY{n+nb}{print}\PY{p}{(}\PY{l+m+mi}{3}\PY{o}{/}\PY{l+m+mi}{2}\PY{p}{)} \PY{c+c1}{\PYZsh{} Division}
        \PY{n+nb}{print}\PY{p}{(}\PY{l+m+mi}{7}\PY{o}{\PYZpc{}}\PY{k}{4}) \PYZsh{} Modulo,The \PYZpc{} operator returns the remainder after division.
        \PY{n+nb}{print}\PY{p}{(}\PY{l+m+mi}{7}\PY{o}{/}\PY{o}{/}\PY{l+m+mi}{4}\PY{p}{)} \PY{c+c1}{\PYZsh{} Floor Division,truncates the decimal without rounding and returns an integer result.}
        \PY{n+nb}{print}\PY{p}{(}\PY{l+m+mi}{2}\PY{o}{*}\PY{o}{*}\PY{l+m+mi}{3}\PY{p}{)} \PY{c+c1}{\PYZsh{} Powers}
        \PY{n+nb}{print}\PY{p}{(}\PY{l+m+mi}{4}\PY{o}{*}\PY{o}{*}\PY{l+m+mf}{0.5}\PY{p}{)} \PY{c+c1}{\PYZsh{} Can also do roots this way}
        \PY{n+nb}{print}\PY{p}{(}\PY{l+m+mi}{3}\PY{o}{+}\PY{l+m+mi}{10}\PY{o}{*}\PY{l+m+mi}{10}\PY{o}{\PYZhy{}}\PY{l+m+mi}{3}\PY{p}{)} \PY{c+c1}{\PYZsh{} Order of Operations followed in Python}
        \PY{n+nb}{print}\PY{p}{(}\PY{p}{(}\PY{l+m+mi}{2}\PY{o}{+}\PY{l+m+mi}{8}\PY{p}{)}\PY{o}{*}\PY{p}{(}\PY{l+m+mi}{13}\PY{o}{\PYZhy{}}\PY{l+m+mi}{3}\PY{p}{)}\PY{p}{)} \PY{c+c1}{\PYZsh{} Can use parentheses to specify orders}
\end{Verbatim}


    \begin{Verbatim}[commandchars=\\\{\}]
{\color{incolor}In [{\color{incolor} }]:} \PY{c+c1}{\PYZsh{} assigning variables}
        \PY{n}{my\PYZus{}dogs} \PY{o}{=} \PY{l+m+mi}{2}\PY{p}{;} \PY{n+nb}{print}\PY{p}{(}\PY{n}{my\PYZus{}dogs}\PY{p}{)}\PY{p}{;} \PY{n+nb}{print}\PY{p}{(}\PY{n+nb}{type}\PY{p}{(}\PY{n}{my\PYZus{}dogs}\PY{p}{)}\PY{p}{)}
        \PY{n}{my\PYZus{}dogs} \PY{o}{=} \PY{p}{[}\PY{l+s+s1}{\PYZsq{}}\PY{l+s+s1}{Sammy}\PY{l+s+s1}{\PYZsq{}}\PY{p}{,} \PY{l+s+s1}{\PYZsq{}}\PY{l+s+s1}{Frankie}\PY{l+s+s1}{\PYZsq{}}\PY{p}{]}\PY{p}{;} \PY{n+nb}{print}\PY{p}{(}\PY{n}{my\PYZus{}dogs}\PY{p}{)}\PY{p}{;} \PY{n+nb}{print}\PY{p}{(}\PY{n+nb}{type}\PY{p}{(}\PY{n}{my\PYZus{}dogs}\PY{p}{)}\PY{p}{)}
\end{Verbatim}


    \begin{Verbatim}[commandchars=\\\{\}]
{\color{incolor}In [{\color{incolor} }]:} \PY{c+c1}{\PYZsh{} re\PYZhy{}use assigned variables}
        \PY{n}{a} \PY{o}{=} \PY{l+m+mi}{5}\PY{p}{;} \PY{n+nb}{print}\PY{p}{(}\PY{n}{a}\PY{p}{)}
        \PY{n}{a} \PY{o}{=} \PY{n}{a} \PY{o}{+} \PY{n}{a}\PY{p}{;} \PY{n+nb}{print}\PY{p}{(}\PY{n}{a}\PY{p}{)}
        \PY{n}{a} \PY{o}{+}\PY{o}{=} \PY{l+m+mi}{10}\PY{p}{;} \PY{n+nb}{print}\PY{p}{(}\PY{n}{a}\PY{p}{)}
        \PY{n}{a} \PY{o}{*}\PY{o}{=} \PY{l+m+mi}{2}\PY{p}{;} \PY{n+nb}{print}\PY{p}{(}\PY{n}{a}\PY{p}{)}
\end{Verbatim}


    \begin{Verbatim}[commandchars=\\\{\}]
{\color{incolor}In [{\color{incolor} }]:} \PY{c+c1}{\PYZsh{} Simple Exercise}
        \PY{n}{my\PYZus{}income} \PY{o}{=} \PY{l+m+mi}{100}
        \PY{n}{tax\PYZus{}rate} \PY{o}{=} \PY{l+m+mf}{0.10}
        \PY{n}{my\PYZus{}tax\PYZus{}amount} \PY{o}{=} \PY{n}{my\PYZus{}income}\PY{o}{*}\PY{n}{tax\PYZus{}rate}
        \PY{n+nb}{print}\PY{p}{(}\PY{n}{my\PYZus{}tax\PYZus{}amount}\PY{p}{)}\PY{p}{;} \PY{n+nb}{type}\PY{p}{(}\PY{n}{my\PYZus{}tax\PYZus{}amount}\PY{p}{)}
\end{Verbatim}


    \subsection{String operations - (str)}\label{string-operations---str}

Strings in Python are actually a ordered sequence. It keeps track of
every element in the string as a sequence. \emph{e.g.} Python
understands the string \texttt{"hello"} to be a sequence of letters in a
specific order. This means we will be able to use \textbf{indexing} to
grab particular letters (like the \emph{first letter}, \emph{last
letter} etc.).

    \begin{Verbatim}[commandchars=\\\{\}]
{\color{incolor}In [{\color{incolor} }]:} \PY{n+nb}{print}\PY{p}{(}\PY{l+s+s1}{\PYZsq{}}\PY{l+s+s1}{hello}\PY{l+s+s1}{\PYZsq{}}\PY{p}{)} \PY{c+c1}{\PYZsh{} Single word, we can also use double quote}
        \PY{n+nb}{print}\PY{p}{(}\PY{l+s+s1}{\PYZsq{}}\PY{l+s+s1}{This is also a string}\PY{l+s+s1}{\PYZsq{}}\PY{p}{)} \PY{c+c1}{\PYZsh{} Entire phrase}
        \PY{n+nb}{print}\PY{p}{(}\PY{l+s+s1}{\PYZsq{}}\PY{l+s+s1}{Use }\PY{l+s+se}{\PYZbs{}n}\PY{l+s+s1}{ to print a new line}\PY{l+s+s1}{\PYZsq{}}\PY{p}{)}
        \PY{n+nb}{print}\PY{p}{(}\PY{l+s+s1}{\PYZsq{}}\PY{l+s+s1}{Use }\PY{l+s+se}{\PYZbs{}t}\PY{l+s+s1}{ to inserts a tab or space into a string}\PY{l+s+s1}{\PYZsq{}}\PY{p}{)}
        \PY{n+nb}{print}\PY{p}{(}\PY{l+s+s1}{\PYZsq{}}\PY{l+s+s1}{See what I mean?}\PY{l+s+s1}{\PYZsq{}}\PY{p}{)}
\end{Verbatim}


    \begin{Verbatim}[commandchars=\\\{\}]
{\color{incolor}In [{\color{incolor} }]:} \PY{c+c1}{\PYZsh{} Be careful with quotes. }
        \PY{l+s+s1}{\PYZsq{}}\PY{l+s+s1}{ I}\PY{l+s+s1}{\PYZsq{}}\PY{n}{m} \PY{n}{using} \PY{n}{single} \PY{n}{quotes}\PY{p}{,} \PY{n}{but} \PY{n}{this} \PY{n}{will} \PY{n}{create} \PY{n}{an} \PY{n}{error}\PY{l+s+s1}{\PYZsq{}}\PY{l+s+s1}{ }
\end{Verbatim}


    The reason for the error above is because the single quote in
\texttt{I\textquotesingle{}m} stopped the string. You can use
combinations of double and single quotes to get the complete statement.

    \begin{Verbatim}[commandchars=\\\{\}]
{\color{incolor}In [{\color{incolor} }]:} \PY{l+s+s2}{\PYZdq{}}\PY{l+s+s2}{Now I}\PY{l+s+s2}{\PYZsq{}}\PY{l+s+s2}{m ready to use the single quotes inside a string!}\PY{l+s+s2}{\PYZdq{}} 
\end{Verbatim}


    \subsubsection{String basics, indexing and
slicing}\label{string-basics-indexing-and-slicing}

\begin{itemize}
\tightlist
\item
  \texttt{{[}{]}} - used after an object to call its \textbf{index}.
  Note that \textbf{indexing} starts at \texttt{0} for Python.
\item
  \texttt{:} - used to perform \textbf{slicing} which grabs everything
  up to a designated point (\texttt{{[}start:stop:step{]}})
\item
  \texttt{len()} - used to check the \textbf{length} of a string.
\end{itemize}

    \begin{Verbatim}[commandchars=\\\{\}]
{\color{incolor}In [{\color{incolor} }]:} \PY{n}{my\PYZus{}string} \PY{o}{=} \PY{l+s+s1}{\PYZsq{}}\PY{l+s+s1}{Hello World}\PY{l+s+s1}{\PYZsq{}} \PY{c+c1}{\PYZsh{} assigning a string}
        
        \PY{c+c1}{\PYZsh{} Shows everything}
        \PY{n+nb}{print}\PY{p}{(}\PY{n}{my\PYZus{}string}\PY{p}{)}
        \PY{n+nb}{print}\PY{p}{(}\PY{n}{my\PYZus{}string}\PY{p}{[}\PY{p}{:}\PY{p}{]}\PY{p}{)}
        
        \PY{n+nb}{print}\PY{p}{(}\PY{n+nb}{len}\PY{p}{(}\PY{n}{my\PYZus{}string}\PY{p}{)}\PY{p}{)} \PY{c+c1}{\PYZsh{} length of the string including spaces}
        \PY{n+nb}{print}\PY{p}{(}\PY{n}{my\PYZus{}string}\PY{p}{[}\PY{l+m+mi}{0}\PY{p}{]}\PY{p}{)} \PY{c+c1}{\PYZsh{} Shows first element }
        \PY{n+nb}{print}\PY{p}{(}\PY{n}{my\PYZus{}string}\PY{p}{[}\PY{l+m+mi}{1}\PY{p}{]}\PY{p}{)} \PY{c+c1}{\PYZsh{} Shows second element}
        \PY{n+nb}{print}\PY{p}{(}\PY{n}{my\PYZus{}string}\PY{p}{[}\PY{o}{\PYZhy{}}\PY{l+m+mi}{1}\PY{p}{]}\PY{p}{)} \PY{c+c1}{\PYZsh{} Shows last letter (one index behind 0 so it loops back around)}
        \PY{n+nb}{print}\PY{p}{(}\PY{n}{my\PYZus{}string}\PY{p}{[}\PY{l+m+mi}{1}\PY{p}{:}\PY{p}{]}\PY{p}{)} \PY{c+c1}{\PYZsh{} Shows everything post the first index}
        \PY{n+nb}{print}\PY{p}{(}\PY{n}{my\PYZus{}string}\PY{p}{[}\PY{p}{:}\PY{l+m+mi}{3}\PY{p}{]}\PY{p}{)} \PY{c+c1}{\PYZsh{} Shows everything up to the 3rd index}
        \PY{n+nb}{print}\PY{p}{(}\PY{n}{my\PYZus{}string}\PY{p}{[}\PY{p}{:}\PY{o}{\PYZhy{}}\PY{l+m+mi}{1}\PY{p}{]}\PY{p}{)} \PY{c+c1}{\PYZsh{} Shows everything but the last letter}
        
        \PY{n+nb}{print}\PY{p}{(}\PY{n}{my\PYZus{}string}\PY{p}{[}\PY{p}{:}\PY{p}{:}\PY{l+m+mi}{1}\PY{p}{]}\PY{p}{)} \PY{c+c1}{\PYZsh{} Shows everything, but jumps in steps = 1}
        \PY{n+nb}{print}\PY{p}{(}\PY{n}{my\PYZus{}string}\PY{p}{[}\PY{p}{:}\PY{p}{:}\PY{l+m+mi}{2}\PY{p}{]}\PY{p}{)} \PY{c+c1}{\PYZsh{} Shows everything, but jumps in steps = 2}
        \PY{n+nb}{print}\PY{p}{(}\PY{n}{my\PYZus{}string}\PY{p}{[}\PY{p}{:}\PY{p}{:}\PY{o}{\PYZhy{}}\PY{l+m+mi}{1}\PY{p}{]}\PY{p}{)} \PY{c+c1}{\PYZsh{} trick to print the string backwards}
\end{Verbatim}


    \subsubsection{String Properties}\label{string-properties}

It's important to note that strings have an important property known as
\textbf{immutability}. This means that once a string is created, the
elements within it can not be changed or replaced.

    \begin{Verbatim}[commandchars=\\\{\}]
{\color{incolor}In [{\color{incolor} }]:} \PY{n}{my\PYZus{}string}\PY{p}{[}\PY{l+m+mi}{0}\PY{p}{]} \PY{o}{=} \PY{l+s+s1}{\PYZsq{}}\PY{l+s+s1}{x}\PY{l+s+s1}{\PYZsq{}} \PY{c+c1}{\PYZsh{} Let\PYZsq{}s try to change the first letter to \PYZsq{}x\PYZsq{}}
\end{Verbatim}


    Notice how the error tells us directly what we can't do, change the item
assignment!

    \begin{Verbatim}[commandchars=\\\{\}]
{\color{incolor}In [{\color{incolor} }]:} \PY{n}{my\PYZus{}string} \PY{o}{=} \PY{n}{my\PYZus{}string}\PY{o}{+}\PY{l+s+s1}{\PYZsq{}}\PY{l+s+s1}{ good morning !!}\PY{l+s+s1}{\PYZsq{}} \PY{c+c1}{\PYZsh{} Concatenate strings}
        \PY{n+nb}{print}\PY{p}{(}\PY{n}{my\PYZus{}string}\PY{p}{)} 
        \PY{n+nb}{print}\PY{p}{(}\PY{l+s+s1}{\PYZsq{}}\PY{l+s+s1}{a}\PY{l+s+s1}{\PYZsq{}}\PY{o}{*}\PY{l+m+mi}{10}\PY{p}{)} \PY{c+c1}{\PYZsh{}  multiplication symbol to create repetition}
\end{Verbatim}


    \subsubsection{String methods}\label{string-methods}

Objects in Python usually have built-in methods. These methods are
functions inside the object that can perform actions or commands on the
object itself i.e.

\texttt{object.method(parameters)}, where \texttt{parameters} are extra
arguments we can pass into the method.

    \begin{Verbatim}[commandchars=\\\{\}]
{\color{incolor}In [{\color{incolor} }]:} \PY{n+nb}{print}\PY{p}{(}\PY{n}{my\PYZus{}string}\PY{o}{.}\PY{n}{upper}\PY{p}{(}\PY{p}{)}\PY{p}{)} \PY{c+c1}{\PYZsh{} Upper Case}
        \PY{n+nb}{print}\PY{p}{(}\PY{n}{my\PYZus{}string}\PY{o}{.}\PY{n}{lower}\PY{p}{(}\PY{p}{)}\PY{p}{)} \PY{c+c1}{\PYZsh{} Lower case}
        \PY{n+nb}{print}\PY{p}{(}\PY{n}{my\PYZus{}string}\PY{o}{.}\PY{n}{split}\PY{p}{(}\PY{p}{)}\PY{p}{)} \PY{c+c1}{\PYZsh{} Split by blank space (default)}
        \PY{n+nb}{print}\PY{p}{(}\PY{n}{my\PYZus{}string}\PY{o}{.}\PY{n}{split}\PY{p}{(}\PY{l+s+s1}{\PYZsq{}}\PY{l+s+s1}{W}\PY{l+s+s1}{\PYZsq{}}\PY{p}{)}\PY{p}{)} \PY{c+c1}{\PYZsh{} Split by a specific (won\PYZsq{}t be included) element}
\end{Verbatim}


    \subsubsection{String formatting}\label{string-formatting}

String formatting lets you inject items into a string rather than trying
to chain items together using commas or string concatenation. e.g. for a
quick comparison refer the code below :

\texttt{student\_name,\ student\_score\ =\ \textquotesingle{}Rakesh\textquotesingle{},\ 70}

\texttt{student\_name+\textquotesingle{}\ scored\ \textquotesingle{}+str(student\_score)+\textquotesingle{}\ points.\textquotesingle{}}
- \textbf{concatenation}

\texttt{f\textquotesingle{}\{student\_name\}\ scored\ \{student\_score\}\ points.\textquotesingle{}}
- \textbf{string formatting}

There are 3 ways to perform string formatting.

\begin{itemize}
\item
  \textbf{Formatting with placeholders} - Oldest method, involves using
  the modulo character \texttt{\%}(referred as the \emph{string
  formatting operator}) to inject strings into your print statements.
\item
  \textbf{Formatting with the \texttt{.format()} method} - A better way
  to format objects into your strings for print statements. Following
  are the advantages:

  \begin{itemize}
  \tightlist
  \item
    Inserted objects can be called by index position.
  \item
    Inserted objects can be assigned keywords.
  \item
    Inserted objects can be reused, avoiding duplication
  \item
    ...
  \item
    Within the curly braces you can assign field lengths, left/right
    alignments, rounding parameters and more
  \item
    You can pass an optional \texttt{\textless{}},\texttt{\^{}}, or
    \texttt{\textgreater{}} to set a left, center or right alignment
  \item
    You can precede the aligment operator with a padding character
  \end{itemize}
\item
  \textbf{Formatted String Literals} (\texttt{f-strings}) - Newest
  method, introduced in Python 3.6, \texttt{f-strings} offer several
  benefits over the older \texttt{.format()} string method described
  above. For one, you can bring outside variables immediately into to
  the string rather than pass them as arguments through
  \texttt{.format(var)}.
\end{itemize}

\textbf{examples}:

    \begin{Verbatim}[commandchars=\\\{\}]
{\color{incolor}In [{\color{incolor} }]:} \PY{c+c1}{\PYZsh{} Formatting with placeholders}
        \PY{n}{student\PYZus{}name}\PY{p}{,} \PY{n}{student\PYZus{}score} \PY{o}{=} \PY{l+s+s1}{\PYZsq{}}\PY{l+s+s1}{Rakesh}\PY{l+s+s1}{\PYZsq{}}\PY{p}{,} \PY{l+m+mi}{70}
        \PY{n+nb}{print}\PY{p}{(}\PY{l+s+s2}{\PYZdq{}}\PY{l+s+s2}{Name of the student is }\PY{l+s+si}{\PYZpc{}s}\PY{l+s+s2}{.}\PY{l+s+s2}{\PYZdq{}} \PY{o}{\PYZpc{}} \PY{n}{student\PYZus{}name}\PY{p}{)} \PY{c+c1}{\PYZsh{} single input}
        \PY{n+nb}{print}\PY{p}{(}\PY{l+s+s2}{\PYZdq{}}\PY{l+s+si}{\PYZpc{}s}\PY{l+s+s2}{ has scored }\PY{l+s+si}{\PYZpc{}s}\PY{l+s+s2}{ in the final exam.}\PY{l+s+s2}{\PYZdq{}} \PY{o}{\PYZpc{}} \PY{p}{(}\PY{n}{student\PYZus{}name}\PY{p}{,} \PY{n}{student\PYZus{}score}\PY{p}{)}\PY{p}{)} \PY{c+c1}{\PYZsh{} multiple input}
\end{Verbatim}


    \begin{Verbatim}[commandchars=\\\{\}]
{\color{incolor}In [{\color{incolor} }]:} \PY{c+c1}{\PYZsh{} Formatting with the .format() method}
        \PY{n+nb}{print}\PY{p}{(}\PY{l+s+s1}{\PYZsq{}}\PY{l+s+s1}{Name of the student is }\PY{l+s+si}{\PYZob{}\PYZcb{}}\PY{l+s+s1}{\PYZsq{}}\PY{o}{.}\PY{n}{format}\PY{p}{(}\PY{n}{student\PYZus{}name}\PY{p}{)}\PY{p}{)}
        \PY{n+nb}{print}\PY{p}{(}\PY{l+s+s1}{\PYZsq{}}\PY{l+s+si}{\PYZob{}0\PYZcb{}}\PY{l+s+s1}{ has scored }\PY{l+s+si}{\PYZob{}1\PYZcb{}}\PY{l+s+s1}{ in the final exam.}\PY{l+s+s1}{\PYZsq{}}\PY{o}{.}\PY{n}{format}\PY{p}{(}\PY{n}{student\PYZus{}name}\PY{p}{,}\PY{n}{student\PYZus{}score}\PY{p}{)}\PY{p}{)} 
\end{Verbatim}


    \begin{Verbatim}[commandchars=\\\{\}]
{\color{incolor}In [{\color{incolor} }]:} \PY{c+c1}{\PYZsh{} Formatting with f\PYZhy{}strings}
        \PY{n+nb}{print}\PY{p}{(}\PY{n}{f}\PY{l+s+s2}{\PYZdq{}}\PY{l+s+s2}{Name of the student is }\PY{l+s+si}{\PYZob{}student\PYZus{}name\PYZcb{}}\PY{l+s+s2}{\PYZdq{}}\PY{p}{)}
        \PY{n+nb}{print}\PY{p}{(}\PY{n}{f}\PY{l+s+s2}{\PYZdq{}}\PY{l+s+si}{\PYZob{}student\PYZus{}name\PYZcb{}}\PY{l+s+s2}{ has scored }\PY{l+s+si}{\PYZob{}student\PYZus{}score\PYZcb{}}\PY{l+s+s2}{ in the final exam.}\PY{l+s+s2}{\PYZdq{}}\PY{p}{)}
\end{Verbatim}


    \subsection{Lists operations - (list)}\label{lists-operations---list}

    \section{Cheat sheets}\label{cheat-sheets}


    % Add a bibliography block to the postdoc
    
    
    
    \end{document}
